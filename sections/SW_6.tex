\section{OFDM}
OFDM has the benefit that frequency-selective fading is no problem and no spectrum is wasted for guard bands. Nevertheless, guard intervals are inserted to prevent ICI(inter channel interference) The following \href{https://youtu.be/i3LBGw8Yle4}{video} might be helpful.
The equalizer is in the frequency domain, and therefore it is much easier.\newline
The total excess delay is the total difference from the first and last signal received through the channel due to reflection. The mean delay of a received signal can be calculated according to \autoref{eq:mean_delay}, where $P_T$ is the power of the transmitted signal and can be calculated according to \autoref{eq:signal_power}. The delay spread can be calculated according to \autoref{eq:rms_delay_spread} and is dependent on \autoref{eq:signal_power} and \autoref{eq:mean_delay}. The delay spread has quite an important role in OFDM, since it defines the guard intervals.
\begin{equation}\label{eq:mean_delay}
\tau_0=\frac{1}{P_T} \sum_k P_k \tau_k
\end{equation}
\begin{equation}\label{eq:rms_delay_spread}
\begin{aligned}
\tau_{\mathrm{RMS}} =\sqrt{\frac{1}{P_T} \sum_k P_k\left(\tau_k-\tau_0\right)^2}
\end{aligned}
\end{equation}
\begin{equation}\label{eq:signal_power}
P_T =\sum_k P_k
\end{equation}
\subsubsection{Exercise Channel Impulse response and Coherence time/bandwidth}
\begin{enumerate}
    \item \textbf{What is the mean delay of a channel with the impulse response of $h(t)=\delta(t)+0.5 \delta\left(t-T_1\right)$ and $T_1=1\mu s$?}
    $$
    \tau_0=\frac{1}{1.25}\left(1^2\cdot \SI{0}{\micro\second}+0.5^2 \cdot \SI{1}{\micro\second}\right)=0.2 \mu \mathrm{s}
    $$
    \item \textbf{What is the spread of the signal mentioned above?}
    $$
    \tau_{\mathrm{RMS}}=\sqrt{\frac{1}{1.25}\left(1^2\cdot(0-0.2)^2+0.5^2 \cdot(1-0.2)^2\right)}=0.4 \mu \mathrm{s}
    $$
    \item \textbf{How does the impulse response look like?}\newline
    (two Dirac pulses, one delayed by T1 and attenuated to half the first one)
    \item \textbf{How would a possible environment look like?} \newline
    \SI{1}{\micro\second} corresponds to around \SI{300}{\meter}, There fore one has a direct path and a object that has a distanc that the signal must travel 300m more than in the direct path.
    \item \textbf{How would the frequency response look like obtained in the Fourier domain?}
    $$
    H(f)=1+0.5 \exp \left(-j 2 \pi f T_1\right)
    $$
    \item\textbf{ Draw the spectrum:}
    \begin{figure}[ht]
        \centering
        \includegraphics[width=13cm]{images/amplitude_spectrum_5.tex}
        \caption{Amplitude Spectrum}
        \label{fig:amp_spec_5}
    \end{figure}
    \begin{figure}[ht]
        \centering
        \includegraphics[width=13cm]{images/amplitude_spectrum_6.tex}
        \caption{Amplitude Spectrum}
        \label{fig:amp_spec_5}
    \end{figure}
    \FloatBarrier
    \item \textbf{Estimate the coherence bandwidth of this channel from the graph sketched} \newline
    Less than 100kHz (smallest bandwidth where the signal changes by 3dB)
    \item \textbf{Compute the coherence bandwidth of this channel and compare it with your estimated value:}\newline
    Depending on the definition we either have
    $$
    B_{\mathrm{coh}}=\frac{1}{2 \pi \tau_{\mathrm{RMS}}}=\frac{1}{2 \pi \cdot 0.4 \mu \mathrm{s}}=398 \mathrm{kHz}
    $$
    or, if the correlation function (among the impulse response at the different frequency positions) is to be $>0.9$
    $$
    B_{\mathrm{coh}}=\frac{1}{50 \tau_{\mathrm{RMS}}}=\frac{1}{50 \cdot 0.4 \mu \mathrm{s}}=50 \mathrm{kHz}
    $$
    \item \textbf{What is the coherence time of this channel?} \newline
    $\infty$, since nothing is moving, environment does not change.
    
\end{enumerate}
\subsubsection{Exercise Coherence time/Coherence Bandwidth}
Given is a wirless system with the following parameters:
\begin{itemize}
    \item Bandwidth w: 20MHz
    \item $f_{RF}$ = \SI{5}{\giga\hertz}
    \item Subcarrier spacing $\Delta f$ = \SI{312.5}{\kilo\hertz}
    \item Symbol time = \SI{3.2}{\micro\second}
    \item Guard interval = \SI{0.8}{\micro\second}
\end{itemize}
\begin{enumerate}
    \item \textbf{Is the coherence time of the channel long enough?} \newline
    From \autoref{eq:ofdm_rule_of_thumb4} and \autoref{eq:doppler_spread} one knows that $T_{\mathrm{coh}} \approx \frac{9}{16 \pi f_d}=\frac{9}{16 \pi \frac{v}{\SI{3e8}{\meter\per\second}}\cdot f_{RF}} \Rightarrow v=\frac{9}{16\cdot \pi \cdot T_{\mathrm{coh}}}\cdot \frac{\SI{3e8}{\meter\per\second}}{f_{RF}}=\frac{9}{16\cdot \pi \cdot \SI{3.2}{\micro\second}}\cdot \frac{\SI{3e8}{\meter\per\second}}{\SI{2.4}{\giga\hertz}}=\SI{3.36e3}{\meter\per\second}$, since $T_{\mathrm{coh}} \gg \frac{N_c}{W}=\frac{1}{\Delta f}=\SI{3.2}{\micro\second}$. Therefore it is long enough (speeds in a room that are larger than \SI{3.36e3}{\meter\per\second} is not common)
    \item \textbf{Is the coherence bandwidth of the channel wide enough?}\newline
    From \autoref{eq:ofdm_rule_of_thumb1} one knows that $B_{\mathrm{coh}} \approx \frac{1}{50 \tau_{\mathrm{RMS}}}$. Furthermore $\tau_{\mathrm{RMS}}=\frac{x_{\mathrm{RMS}}}{c_0}$ (the delay spread is the length spread of the different paths divided by the speed of light). With \autoref{eq:N_C_selection} one can write $B_{\mathrm{coh}} \approx \frac{1}{50 \tau_{\mathrm{RMS}}} \gg \frac{W}{N_C}=\frac{1}{\Delta f}=\SI{312.5}{\kilo\hertz} \Rightarrow \frac{c_0}{50 x_{\mathrm{RMS}}} \gg  = \SI{312.5}{\kilo\hertz} \Rightarrow \frac{\SI{3e8}{\meter\per\second}}{\SI{312.5}{\kilo\hertz} \cdot 50} \gg x_{\mathrm{RMS}} \Rightarrow \SI{19.2}{\meter} \gg x_{\mathrm{RMS}}$ which is plausible for a normal room.
\end{enumerate}

\begin{equation}\label{eq:ofdm_coherence}
T_{\text {coh }} \propto \frac{1}{f_d}
\end{equation}
\begin{equation}
B_{\mathrm{coh}} \propto \frac{1}{\tau_{\text {RMS }}}
\end{equation}
\begin{itemize}
    \item $B_{coh}$ (Coherence bandwidth) = bandwidth over which the channel can be considered constant. Note: $T_{coh}$ is completely independent of $B_{coh}$
    \item $T_{coh}$ (Coherence time) = time over which the channel can be considered constant. Has to do with the motion dynamics.
    \item $f_d$=Doppler spread (when we head to the transmission station with our phone, we get a doppler effect, $f_{RF}$ (center frequency) gets higher, doppler spread is normally in the range of Hz (a few hertz)), see also \autoref{eq:doppler_spread}
    \begin{equation}\label{eq:doppler_spread}
        f_d=\frac{V}{C_0}\cdot f_{RF}\cdot cos(\alpha)
    \end{equation}
    \item $B$=Bandwidth
    \item $T$=symbol time
    \item $N_c$ = Number of subcarriers
    \item $\Delta f$ = subcarrier spacing
\end{itemize}



$$
\begin{array}{lll}
\hline \text { fading type } & \text { condition } & \begin{array}{l}
\text { alternative formulation } \\
\text { of condition }
\end{array} \\
\hline \text { flat fading } & B_{\mathrm{coh}} \gg B & \tau \ll T \\
\text { frequency-selective fading } & B_{\mathrm{coh}}<B & \tau>T \\
\hline \text { slow fading } & T_{\mathrm{coh}} \gg T & f_d \ll B \\
\text { fast fading } & T_{\mathrm{coh}}<T & f_d>B \\
\hline
\end{array}
$$



The formula below is in our signal
\begin{equation}
B \propto \frac{1}{T}
\end{equation}
It's important to have the following in mind: When one has a small bandwidth the signal gets long (for a sharp impulse one needs a huge bandwidth (UWB)). One now has to find a number of sub carriers $N_c$ that make the bandwidth not to small ($N_c$ not to large) $\Rightarrow$, otherwise the time gets to long (problems with moving objects) and do not make them to wide ($N_c$ not to small)$\Rightarrow$, otherwise frequency fading could cause issues. The behaviour can be described with \autoref{eq:N_C_selection}.
\begin{equation}\label{eq:N_C_selection}
\frac{W}{B_{\mathrm{coh}}} \ll N_c \ll W \cdot T_{\mathrm{coh}}
\end{equation}
Furthermore one must be aware of the following Rule of Thumbs: \autoref{eq:ofdm_rule_of_thumb1} can be used if the frequency correlation function is above 0.9.
\begin{equation}\label{eq:ofdm_rule_of_thumb1}
B_{\mathrm{coh}} \approx \frac{1}{50 \tau_{\mathrm{RMS}}}
\end{equation}

\autoref{eq:ofdm_rule_of_thumb2} can be used if the frequency correlation function is above 0.5.
\begin{equation}\label{eq:ofdm_rule_of_thumb2}
B_{\mathrm{coh}} \approx \frac{1}{5 \tau_{\mathrm{RMS}}}
\end{equation}

\autoref{eq:ofdm_rule_of_thumb3} is sometimes also used (leading to yet another value for the correlation function).
\begin{equation}\label{eq:ofdm_rule_of_thumb3}
B_{\mathrm{coh}} \approx \frac{1}{2 \pi \tau_{\mathrm{RMS}}}
\end{equation}

Very similar rules can be found for the relationship given by \autoref{eq:ofdm_coherence}(for the time correlation to be above 0.5), see also \autoref{eq:doppler_spread} for the Doppler spread . 
\begin{equation}\label{eq:ofdm_rule_of_thumb4}
T_{\mathrm{coh}} \approx \frac{9}{16 \pi f_d}
\end{equation}
With those equations one could make the following statement for coherence times of about 0.5.
$$
W \cdot 5\cdot \tau_{\mathrm{RMS}}  \ll N_c \ll \frac{W \cdot 9}{16 \cdot \pi \cdot \frac{v}{\SI{3e8}{\meter\per\second}}\cdot f_{RF}}
$$
\paragraph{Exercise: For 802.11a (WLAN), we have W = 20MHz and Nc = 64. What is the subcarrier distance? How large should $B_{coh}$ and $T_{coh}$ of the channel be at least?}\mbox{}\newline
The subcarrier distance is:
$$
\frac{W}{N_c}=\frac{20 \mathrm{MHz}}{64}=312.5 \mathrm{kHz}
$$
$$
\begin{aligned}
&B_{\mathrm{coh}} \gg \frac{W}{N_c}=312.5 \mathrm{kHz}, \quad \text { e.g., } 3.125 \mathrm{MHz} \\
&T_{\mathrm{coh}} \gg \frac{1}{312.5 \mathrm{kHz}}=3.2 \mu \mathrm{s}, \quad \text { e.g., } 32 \mu \mathrm{s}
\end{aligned}
$$

FFT is a circular and not a linear convolution.
$$
\begin{aligned}
\frac{9}{16 \pi f_d}=T_{c o h} & \gg \frac{N_c}{W}=\frac{1}{\Delta f}=\frac{1}{312.5 k \mathrm{kz}}=3.2 \mu \mathrm{s} \\
f d & \ll \frac{9}{16 \pi \cdot 3.2 \mu s} \\
v &<\frac{9}{16 \pi \cdot 3.2 \mu \mathrm{s}} \cdot \frac{C_0}{f \mathrm{RF}}=3.6 \cdot 10^3 \mathrm{~m} / \mathrm{s}
\end{aligned}
$$

In OFDM we have a lot of sub carriers, and all sub carriers are orthogonal.
\subsection{Water filling}
\begin{itemize}
    \item $N_c$ =Number of sub carriers
    \item $W$ = Total bandwidth available
    \item $S$ = Total signal power
    \item $S_n$ = Received power in the n'th channel
    \item $C$ = Shannon Capacity
    \item $\eta$ = noise power density
    \item $H_n$ = Transfer function of a single sub carrier
\end{itemize}
Shannon's channel capacity:
\begin{equation}\label{eq:shannon_capacity}
C=W \log _2\left(1+\frac{S}{N}\right)=W \log _2\left(1+\frac{S}{\eta W}\right)
\end{equation}
In a multicarrier system:
$$
C=\sum_{n=1}^{N_c} \frac{W}{N_c} \log _2\left(1+\frac{S_n}{\eta W / N_c}\right)
$$
Power Received:
\begin{equation}\label{eq:attenuation_of_subcarrier}
S_n=\underbrace{P_n}_{\text{power allocated in the n'th subcarrier}} \cdot\left|H_n\right|^2
\end{equation}
\begin{equation}\label{eq:waterfilling_beta}
\beta=\frac{P+\eta W / N_c \sum_{n=1}^{N_c} \frac{1}{\left|H_n\right|^2}}{N_c}
\end{equation}
\begin{equation}\label{eq:waterfilling_beta2}
P_n=\beta-\frac{\eta W / N_c}{\left|H_n\right|^2}, \quad n=1 \ldots N_c
\end{equation}
\paragraph{Example: Imagine a two-carrier system with total power P = $P_1$ + $P_2$. If the two channel coefficients $H_1$ and $H_2$ are given, derive the choice of $P_1$ and $P_2$ that optimize the use of the channels.}
With \autoref{eq:shannon_capacity} and \autoref{eq:attenuation_of_subcarrier} one can create the following equation system:
$$
C=\frac{W}{2} \log _2\left(1+\frac{P_1 \cdot\left|H_1\right|^2}{\eta W / 2}\right)+\frac{W}{2} \log _2\left(1+\frac{P_2 \cdot\left|H_2\right|^2}{\eta W / 2}\right)
$$
Since we want to have it's maximum, one has to derive it and then set it to zero.\newline
With some magic we end then with the following:
$$
P_1=\frac{\eta W}{4} \cdot \frac{\left|H_1\right|^2-\left|H_2\right|^2}{\left|H_1\right|^2\left|H_2\right|^2}+\frac{P}{2}
$$
When $P_1$ is negative, we do not assign any energy to it.
\paragraph{The DAB system (digital audio broadcast) facilitates different OFDM Transmission modes to allow for some
flexibility with respect to environment and change of environment (due to vehicle movement). Assume a total
bandwidth of W = 1.536 MHz.}
\textbf{Complete the following table:}
$$
\begin{array}{lcccr}
\hline \text { Transm. mode } & \text { No. of subcarriers } & \text { Subcarrier spacing } & \text { Symbol time }{ } T_S & \text { Guard time } \\
\hline \text { TM I } & 1536 & 1 \mathrm{kHz} & 1.246 \mathrm{~ms} & 246 \mu \mathrm{s} \\
\text { TM IV } & 768 & 2 \mathrm{kHz} & 623 \mu \mathrm{s} & 123 \mu \mathrm{s} \\
\text { TM II } & 384 & 4 \mathrm{kHz} & 312 \mu \mathrm{s} & 62 \mu \mathrm{s} \\
\text { TM III } & 192 & 8 \mathrm{kHz} & 156 \mu \mathrm{s} & 31 \mu \mathrm{s} \\
\hline
\end{array}
$$
\textbf{What efficiency (with respect to the guard time 'wasted') does the OFDM system have?}
$$
\eta=\frac{1.246-0.246}{1.246}=80 \%
$$
\textbf{State the numerical condition for the coherence bandwidth for TM I}
According to \autoref{eq:N_C_selection}:
$$
B_{\mathrm{coh}} \gg \frac{1.5 \mathrm{MHz}}{1536} \approx 1 \mathrm{kHz}
$$
\textbf{With the rule of thumb that the delay spread and the coherence bandwidth are related through state the condition for the delay spread for TM I.}
$$
\tau_{\mathrm{RMS}}=\frac{1}{2 \pi B_{\mathrm{coh}}}
$$
one gets:
$$
\tau_{\mathrm{RMS}} \ll 160 \mu \mathrm{s}
$$
\textbf{State the numerical condition for the coherence time for TM I.}
According to \autoref{eq:N_C_selection}:
$$
T_{\text {coh }} \gg \frac{1536}{1.5 \mathrm{MHz}} \approx 1 \mathrm{~ms}
$$
\textbf{With the rule of thumb that (for a correlation coefficient of 0.5) the coherence time is in relation with the Doppler frequency shift is as in the euqation below. Compute the coherence time for a vehicle speed of 120 km/h and an RF of 1.5 GHz (L-band as used in DAB).}
$$
T_{\mathrm{coh}} \approx \frac{9}{16 \pi f_D}
$$
The Doppler frequency can be calculated according to \autoref{eq:doppler}
\begin{equation}\label{eq:doppler}
f_D=\frac{v}{c} \cdot f_{\mathrm{RF}}
\end{equation}
$$
T_{\mathrm{coh}} \approx \frac{9}{16 \pi \frac{v}{c} \cdot f_{\mathrm{RF}}}=1.1 \mathrm{~ms}
$$
\paragraph{Is the coherence-time condition satisfied sufficiently for TM I?}.
TM I is suitable for large delays but slow vehicle speed. TM III is better suited to high-speed, but allows
only moderate delay spreads.
\subsection{peak-to-average-power Problem (PAPR)} 
\begin{figure}[ht]
    \centering
    \includegraphics[width=13cm]{images/papr.tex}
    \caption{Different Subcarriers}
    \label{fig:papr}
\end{figure}
As one can see in \autoref{fig:papr} the average energy ($P_{avg}$) of the sub carriers is $N_c \cdot \underbrace{x_n^2}_{P_{sub}}$. But the peak energy $P_{Peak}$ is $(N_c \cdot x_n)^2$, which is much higer. Therefore it is possible that one has some very high peaks in an OFDM system.
\subsubsection{Example PAPR}
\paragraph{Imagine an OFDM system with 8 subcarriers of BPSK modulated signals, each with power Psub. The absolute
highest peak possible for such a signal can therefore be 8 times the value of one isolated subcarrier.}
\begin{enumerate}
    \item \textbf{Compute the average power, the peak power and thus the peak-to-average-power value.}
    $$
\begin{aligned}
P_{\mathrm{avg}} &=N P_{\mathrm{sub}}=8 P_{\mathrm{sub}} \\
P_{\text {peak }} &=N^2 P_{\mathrm{sub}}=64 P_{\text {sub }} \\
\mathrm{PAPR} &=\frac{P_{\text {peak }}}{P_{\mathrm{avg}}}=8
\end{aligned}
$$
    \item \textbf{If we have one redundant channel, i.e., we can choose the eighth channel to reduce the amplitude of the
composite signal, what reduction in the peak-to-average-power value would we get?}\newline
The idea is to sent a minus one when all other channels send a one and vice versa, therefore the max amplitude reduces to $6\cdot x_{max}$
$$
\begin{aligned}
P_{\mathrm{avg}} &=8 P_{\mathrm{sub}} \\
P_{\text {peak }} &=36 P_{\text {sub }} \\
\text { PAPR } &=\frac{P_{\text {peak }}}{P_{\text {avg }}}=4.5
\end{aligned}
$$
\item \textbf{Very often, not the highest peak is considered, but the percentage that a certain amplitude is exceeded.
Compute for the original 8-subcarrier OFDM system, the probability that the signal exceeds 3 times the
amplitude of a single subcarrier.}\newline
On the TI-Nspire you find the binomial coefficient under menu$\rightarrow$5:Probability $\rightarrow$3: Combinations $\Rightarrow$ nCr\newline
Furthermore one knows from \autoref{eq:binomial_function} that the following holds: 
$$
    f(x)=P(X=x)=\left(\begin{array}{c}
    n \\
    x
    \end{array}\right) p^x \cdot q^{n-x} \quad(x=0,1,2, \ldots, n)
$$
Therefore
$$
P(X\leq6)=\underbrace{2}_{\text{since all negative is also a soluiton}} \cdot\sum_{k=6}^8\left(\begin{array}{l}
8 \\
k
\end{array}\right)\left(\frac{1}{2}\right)^8\cdot\left(\frac{1}{2}\right)^{8-k}=2 \frac{1}{256}(1+8+28)=\frac{74}{256}=29 \%
$$
since the amplitude is larger than 3 when 6,7 or 8 subcarriers show a one (when only 5 subcarriers show a one one has 5-3=2, which is less than one).
$$
p=2 \sum_{k=0}^2\left(\frac{1}{2}\right)^8\left(\begin{array}{l}
8 \\
k
\end{array}\right)=2 \frac{1}{256}(1+8+28)=\frac{74}{256}=29 \%
$$
\item \textbf{Compute for the OFDM system with one subcarrier chosen such as to lower the amplitude, the probability
that the signal exceeds 3 times the amplitude of a single subcarrier.}
$$
p=2 \sum_{k=0}^1\left(\frac{1}{2}\right)^7\left(\begin{array}{l}
7 \\
k
\end{array}\right)=2 \frac{1}{128}(1+7)=\frac{16}{128}=12.5 \%
$$
\end{enumerate}
\subsection{Diversity}