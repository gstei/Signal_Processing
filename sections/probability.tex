\section{Formulas}
\subsection{Probability}
\subsubsection{Q-funciton}
\begin{equation}\label{q-funciton}
Q(x)=\frac{1}{\sqrt{2 \pi}} \int_x^{\infty} \exp \left(-\frac{u^2}{2}\right) d u
\end{equation}
\subsubsection{Binomialverteilung}
Ein Bernoulli-Experiment mit den beiden sich gegenseitig ausschliessenden Ergebnissen (Ereignissen) $A$ und $\bar{A}$ werde $n$-mal nacheinander ausgefuehrt (sog. mehrstufiges Bernoulli-Experiment vom Umfang $n$ ). Dann genuegt die diskrete Zufallsvariable
$$
    \text{$X=$ Anzahl der Versuche, in denen das Ereignis A eintritt}
$$
der sog. Binomialverteilung mit der Wahrscheinlichkeitsfunktion
\begin{equation}\label{eq:binomial_function}
    f(x)=P(X=x)=\left(\begin{array}{c}
    n \\
    x
    \end{array}\right) p^x \cdot q^{n-x} \quad(x=0,1,2, \ldots, n)
\end{equation}

und der zugehoerigen Verteilungsfunktion
$$
F(x)=P(X \leq x)=\sum_{k \leq x}\left(\begin{array}{l}
n \\
k
\end{array}\right) p^k \cdot q^{n-k} \quad(x \geq 0)
$$
(fuer $x<0$ ist $F(x)=0$ ). $n$ und $p$ sind dabei die Parameter der Verteilung. Die Kennwerte oder Masszahlen der Binomialverteilung lauten:
\begin{itemize}
    \item Mittelwert: $\mu=n p$
    \item Varianz: $\quad \sigma^2=n p q=n p(1-p)$
    \item Standardabweichung: $\sigma=\sqrt{n p q}=\sqrt{n p(1-p)}$
\end{itemize}
Dabei bedeuten:
\begin{itemize}
    \item p: Konstante Wahrscheinlichkeit für das Eintreten des Ereignisses A beim Einzelversuch $(0<p<1)$
    \item $q$ : Konstante Wahrscheinlichkeit für das Eintreten des zu A komplementären Ereignisses $\bar{A}$ beim Einzelversuch $(q=1-p)$
    \item $n$ : Anzahl der Ausführungen des Bernoulli-Experiments (Umfang des mehrstufigen Bernoulli-Experiments)
\end{itemize}
