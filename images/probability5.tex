% \begin{figure}
% \begin{tikzpicture}[
%         line width=1pt, &gt;=latex, scale=0.8,
%         sb/.style={text width=1.5cm, align=center,inner sep=0pt}]
\documentclass[tikz,14pt,border=10pt,subpreambles=true]{standalone}
\newcommand{\DSB}[1]{ 
  \def\xa{3} \def\ya{0}
    \def\xb{1} \def\yb{2}
    \def\xc{0} \def\yc{1.7}
    \def\tang{.4}
    %   
    \def\curve{
            (#1-\xa,\ya) .. controls ++ (2*\tang,0) and ++ (-\tang,0) ..
            (#1-\xb,\yb) .. controls ++ (.5*\tang,0) and ++ (-\tang,0) ..
            (#1,\yc) .. controls ++ (\tang,0) and ++ (-.5*\tang,0) ..
            (#1+\xb,\yb) .. controls ++ (\tang,0) and ++ (-2*\tang,0) ..
            (#1+\xa,\ya)
            }
    \path[fill=violet!40] (#1-\xa,0) -- \curve -- (#1+\xa,0) -- cycle;
    \draw  \curve;
    }
\begin{document}
\begin{tikzpicture}[line width=1pt, >=latex, scale=0.8, sb/.style={text width=1.5cm, align=center,inner sep=0pt}]
    \DSB{0} 
    \draw   (-10,0) -- (10,0) node [below left] {$f \longrightarrow$}
            (0,0) node[below] {0} --++ (0,3) node [below right] {$M(f)$};
    \path   (-3,0) node[below] {$-B$} -- (3,0) node[below] {$B$};
    \node[left] at (10,.5) {(a) Baseband};
    
    \begin{scope}[yshift=-5cm]
        \def\xs{5}
        \DSB{-\xs} \DSB{\xs}
            
        \draw   (-10,0) -- (10,0) node [below left] {$f \longrightarrow$}
                (0,0) node[below] {0} --++ (0,3)
                (-\xs,0) node[below] {$-f_C$} --++ (0,2.5)
                (\xs,0) node[below] {$f_C$} --++ (0,2.5);
        \node[left] at (10,.5) {(b) DSB};
        
        \draw[stealth-] (-\xs-1.5,1.5) to [bend right] ++ (-1,.5) node[left,sb] {Upper sideband};
        \draw[stealth-] (-\xs+1.5,1.5) to [bend left] ++ (1,.5) node[right,sb] {Lower sideband};
        
        \draw[stealth-] (\xs-1.5,1.5) to [bend right] ++ (-1,.5) node[left,sb] {Lower sideband};
        \draw[stealth-] (\xs+1.5,1.5) to [bend left] ++ (1,.5) node[right,sb] {Upper sideband};
    \end{scope} 
                
    \end{tikzpicture}
\end{document}