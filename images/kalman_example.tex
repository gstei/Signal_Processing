%%%%%%%%%%%%%%%%%%%%%%%%%%%%%%%%%%%%%%%%%%%%%%%%%%%%%%%%%%%%%%%
%
% Welcome to Overleaf --- just edit your LaTeX on the left,
% and we'll compile it for you on the right. If you open the
% 'Share' menu, you can invite other users to edit at the same
% time. See www.overleaf.com/learn for more info. Enjoy!
%
% Note: you can export the pdf to see the result at full
% resolution.
%
%%%%%%%%%%%%%%%%%%%%%%%%%%%%%%%%%%%%%%%%%%%%%%%%%%%%%%%%%%%%%%%

% Title: Block diagram of Third order noise shaper in Compact Disc Players
% Author: Ramón Jaramillo
\documentclass[tikz,14pt,border=10pt,subpreambles=true]{standalone}
\usepackage{textcomp}
\usetikzlibrary{shapes,arrows}
\usepackage{amsmath,amssymb,tikz}
\begin{document}
% Definition of blocks:
\tikzset{%
  block/.style    = {draw, thick, rectangle, minimum height = 3em,
    minimum width = 3em},
  sum/.style      = {draw, circle, node distance = 2cm}, % Adder
  input/.style    = {coordinate}, % Input
  output/.style   = {coordinate}, % Output
  amp/.style = {regular polygon, regular polygon sides=3,
              draw, fill=white, 
              inner sep=0.5mm, outer sep=0mm,
              shape border rotate=90},
  bullet/.style={circle, fill, minimum size=2pt, inner sep=2pt, outer sep=0pt},
  bullet2/.style={circle, fill, minimum size=2pt, inner sep=2pt, outer sep=0pt,fill=gray!50}
}
% Defining string as labels of certain blocks.
\newcommand{\suma}{\Large$+$}
\newcommand{\inte}{$\displaystyle \int$}
\newcommand{\derv}{\huge$\frac{d}{dt}$}

\begin{tikzpicture}[auto, thick, node distance=2cm, >=triangle 45]
\draw
	% Drawing the blocks of first filter :
	node [input, bullet2, label=left:{$x[k]$}] (input1) {}


	node [sum, right of=input1] (suma1) {\suma}
    node [block, right of=suma1] (delay) {$z^{-1}$}
    node [amp, below of=delay] (gain) {\Large$\textcolor{red}{\frac{1}{2}}$}
    node [right of=delay, label=above:{$x[k]$},bullet] (point1) {}
    node [right of=point1,sum] (suma2) {\suma}
    node [above of=suma2,bullet2,label=above:{$v[k]$}] (noise) {}
    node [right of=suma2, bullet2, label=right:{$y[k]$}] (output) {}
     ;
    % Joining blocks. 
    % Commands \draw with options like [->] must be written individually
	\draw[->](input1) -- node {}(suma1);
	\draw[->](suma1) -- node {$x_{k+1}$} (delay);
	\draw[->](suma2) -- node {} (output);
	\draw[->](delay) -- node {} (suma2);
	\draw[->](noise) -- node {} (suma2);
	\draw[->](point1) |- node[near end]{} (gain);
	\draw[->](gain) -| node {} (suma1);
	% Adder

	
\end{tikzpicture}
\end{document}