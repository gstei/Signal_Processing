% \href{https://tikz.net/gauss6ians/}{\begin{tikzpicture}
\documentclass[tikz,14pt,border=10pt,subpreambles=true]{standalone}
\usepackage{amsmath} % for \dfrac
\tikzset{>=latex} % for LaTeX arrow head
\usepackage{pgfplots} % for the axis environment
% \usepackage{xcolor}
\usepackage[outline]{contour} % halo around text
\contourlength{1.2pt}
\usetikzlibrary{positioning,calc}
\usetikzlibrary{backgrounds}% required for 'inner frame sep'
%\usepackage{adjustbox} % add whitespace (trim)
% define gauss6ian pdf and cdf


\pgfmathdeclarefunction{gauss6}{3}{%
  \pgfmathparse{1/(#3*sqrt(2*pi))*exp(-((#1-#2)^2)/(2*#3^2))}%
}


\colorlet{mydarkblue}{blue!30!black}
% to fill an area under function
\usepgfplotslibrary{fillbetween}
\usetikzlibrary{patterns}
\pgfplotsset{compat=1.12} % TikZ coordinates &lt;-&gt; axes coordinates
% https://tex.stackexchange.com/questions/240642/add-vertical-line-of-equation-x-2-and-shade-a-region-in-graph-by-pgfplots
% plot aspect ratio
%\def\axisdefaultwidth{8cm}
%\def\axisdefaultheight{6cm}
% number of sample points
\def\N{50}
\begin{document}
\begin{tikzpicture}
\message{Normal distributions, different mu^^J}
  
  \def\q{-0.856};
  \def\B{0};
  \def\S{1};
  \def\Bs{1.0};
  \def\Ss{1.0};
  \def\xmax{\S+3.2*\Ss};
  \def\ymin{{-0.15*gauss6(\B,\B,\Bs)}};
  
  \begin{axis}[every axis plot post/.append style={
               mark=none,domain={-1*(\xmax)}:{1.08*\xmax},samples=\N,smooth},
               xmin={-1*(\xmax)}, xmax=\xmax,
               ymin=\ymin, ymax={1.1*gauss6(\B,\B,\Bs)},
               axis lines=middle,
               axis line style=thick,
               enlargelimits=upper, % extend the axes a bit to the right and top
               % ticks=none,
               xlabel=$x$,
               x label/.style={at={(current axis.right of origin)},anchor=north west},
               width=0.7*\textwidth, height=0.5*\textwidth,
               clip=false, % prevent labels falling off
               y=200pt
              ]
    
    % PLOTS
    \addplot[blue, name path=B,thick] {0.2*gauss6(x,\B,\Bs)};
    \addplot[red,  name path=S,thick] {0.8*gauss6(x,\S,\Ss)};
    \addplot[black,dashed,thick]
      coordinates {(\B, {0.202*gauss6(\B,\B,\Bs)}) (\B,{-0.05*gauss6(\B,\B,\Bs)})}
      node[below=-4pt] {\strut$0$};
    \addplot[black,dashed,thick]
      coordinates {(\S, {0.802*gauss6(\S,\S,\Ss)}) (\S,{-0.05*gauss6(\S,\S,\Ss)})}
      node[below=-4pt] {\strut$1$};
    \addplot[black,dashed,thick]
      coordinates {(\q, {0.20*gauss6(\B,\B,\Bs)}) (\q,{-0.05*gauss6(\B,\B,\Bs)})}
      node[below=2pt] {\strut$x_\text{cut}$};
    
    % FILL
    \path[name path=xaxis]
      (\q,0) -- (\pgfkeysvalueof{/pgfplots/xmax},0); %\pgfkeysvalueof{/pgfplots/xmin}
    \addplot[white!50!blue] fill between[of=xaxis and B, soft clip={domain=\q:\xmax}];
    %\addplot[white!50!red]  fill between[of=xaxis and S, soft clip={domain=-1*\xmax:0}];
    
    % LABELS
    \node[above=8pt,left,black!20!blue] at (   \B,  {gauss6(\B,\B,\Bs)}) {$p(z \mid X=0)\cdot p(X=0)$};
    \node[above=8pt,right,black!20!red]  at (1.05*\S,{gauss6(\S,\S,\Ss)}) {$p(z \mid X=1)\cdot p(X=1)$};
    \node[right,black!20!blue,scale=1.3] at ({0.1*\q},{0.2*gauss6(1.0*\q,\B,\Bs)}) {\strut$\alpha$};
    
  \end{axis}
\end{tikzpicture}
\end{document}